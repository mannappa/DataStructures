\documentclass{article}
\usepackage[utf8]{inputenc}

\usepackage{geometry}
 \geometry{
 a4paper,
 total={170mm,257mm},
 left=30mm,
 right=30mm,
 top=20mm,
 }

 \usepackage{mathtools}
 \usepackage{verbatim}
 \usepackage{algorithm}
 \usepackage[noend]{algpseudocode}
\usepackage[utf8]{inputenc}
\begin{document}
{
\centering \title*{\huge \textbf Analysis of Important Data Structures} 
}
\section{Singly and Doubly Linked List}
\subsection{Operators Allowed}
\begin{itemize}
\renewcommand{\labelitemi}{$\bullet$}
    \item Insert at Front
    \item Insert at Back
    \item Insert at Position
    \item Search
    \item Delete with Node Value Given
    \item Delete with Position
    \item Display
\end{itemize}
\subsection{C++ Implementation}
\begin{itemize}
\renewcommand{\labelitemi}{$\bullet$}
    \item Singly Linked List Git-Location : DataStructures/source/SinglyLinkedList.cpp
    \item Double Linked List Git-Location : DataStructures/source/DoublyLinkedList.cpp
\end{itemize}
\subsection{Time and Space Complexity Analysis}
\begin{tabular}{c c c c c c c}
\hline\hline 
Method & Time(Avg) & Time(Best) & Time(Worst) & Space(Avg) & Space(Best) & Space(Worst) \\
\hline
Insert & O(1) & O(n) & O(n) & constant & constant & constant\\
Delete & O(n) & O(1) & O(n) & constant & constant & constant \\
Search & O(n) & O(1) & O(n) & constant & constant & constant \\
Form-n-List & O(n) & O(n) & O(n) & O(n) & O(n) & O(n) \\
\hline %inserts single line
\end{tabular}

\section{Binary Tree Types}
\begin{itemize}
\renewcommand{\labelitemi}{$\bullet$}
    \item \textbf{Full/Proper/Plane/Strictly BT} : A full binary tree (sometimes referred to as a proper or plane binary tree) is a tree in which every node in the tree has either 0 or 2 children.
    \item \textbf{Complete BT} : A complete binary tree is a binary tree in which every level, except possibly the last, is completely filled, and all nodes are as far left as possible.
\end{itemize}

\section{Binary Search Tree(BST)}
\subsection{C++ Implementation}
\begin{itemize}
\renewcommand{\labelitemi}{$\bullet$}
    \item Binary Search Tree Git-Location : DataStructures/source/BinarySearchTree.cpp
\end{itemize}
\subsection{Time and Space Complexity Analysis}
\begin{tabular}{c c c c c c c}
\hline\hline 
Method & Time(Avg) & Time(Best) & Time(Worst) & Space(Avg) & Space(Best) & Space(Worst) \\
\hline
Insert & O($log_{2}$n) & O(1) & O(n)\textsuperscript{1} & constant & constant & constant\\
Delete & O($log_{2}$n) & O(1) & O(n)\textsuperscript{1}  & constant & constant & constant \\
Search & O($log_{2}$n) & O(1) & O(n)\textsuperscript{1} & constant & constant & constant \\
Form-n-BST & O(n$log_{2}$n) & O(n$log_{2}$n) & O(n)\textsuperscript{1} & O(n) & O(n) & O(n) \\
\hline
\end{tabular}
\begin{itemize}
\renewcommand{\labelitemi}{$\bullet$}
    \item note-1 : If the BST is formed in the worst way such that all the elements are either on the right/left of every node (8->9->10).
\end{itemize} 
\end{document}
